\documentclass{report}
\usepackage{fullpage}
\usepackage[utf8]{inputenc}
\renewcommand{\baselinestretch}{2}
\author{Your Name Here}
\title{Your Title Here}
\begin{document}
\maketitle
tableofcontents




\section{Introduction}
Ce document rapporte mon activité durant le stage de fin d'études que j'ai 
realisé a l'Inria, l'Institut National de Recherche en Informatique et
Automatique, au sein de l'équipe CORSE (Compilation and Optimisation in Runtime).
J'ai donc eu affaires naturellement à des problématiques de recherche mais aussi à
des problématiques d'ingénierie tout au long de ces six mois. Plus précisément, mes travaux ont été
liées à des problématiques de programmation concurrente et multithreadés. Il s'est agi au départ d'évaluer
le comportement et le partage des ressources sur des architectures dites NUMA, Non-Uniform Memory Access. 
La problématique a ensuite évolué, pour des raisons que l'on verra, vers des questions de partage de cache
entre différents threads.
En l'état, mon travail a consisté en la réalisation d'un benchmark permettant d'évaluer les performances
d'une architecture lorsque qu'un ou plusieurs threads sont limités et se disputent des ressources mémoire.
On expliquera ici les difficultés qui ont été rencontrées et comment ces difficultés ont pu être dépassées.
La problématique a ensuite été d'essayer de parvenir à proposer des algorithmes pertinents pour 

\chapter{l'entreprise}

L'Inria est l'Institut National de recherche en Informatique et Automatique

\end{document}
